% resume.tex
%
%
\documentclass[11pt]{article}
\usepackage{fullpage1}	
\usepackage{atbeginend}
\usepackage[pdftex]{hyperref}
\textheight=8.7in
\pagestyle{empty}
\raggedbottom
\raggedright
\setlength{\tabcolsep}{0in}
\newenvironment{itemize*}%
  {\begin{itemize}%
    \setlength{\itemsep}{0pt}%
    \setlength{\parskip}{0pt}%
	}
  {\end{itemize}}
\newenvironment{description*}%
  {\begin{description}%
   \setlength{\itemsep}{0pt}%
   \setlength{\parskip}{0pt}%
	}
  {\end{description}}
\begin{document}

\begin{tabular*}{6.5in}{l@{\extracolsep{\fill}}r}
\textbf{\large{Rahul Garg}}  & \\
Department of Computer Science and Engineering &  Ph:(206) 708 5975\\
University of Washington & \href{http://www.cs.washington.edu/homes/rahul}{http://www.cs.washington.edu/homes/rahul}\\ 
Box 352350, Seattle, WA 98195 &  email: \href{mailto:rahul@cs.washington.edu}{rahul@cs.washington.edu}\\
\end{tabular*}
\\
\vspace{0.05in}
\rule{6.5in}{2pt}
\\
%\vspace{0.10in}
%{\large \textbf{Objective}}
%\\
%\vspace{0.10in}
%Working in a challenging environment involving research in Computer Vision and Graphics

%\\
%\vspace{0.10in}
%\rule{6.5in}{2pt}
%\\
\vspace{0.10in}
{\large \textbf{Education}}

	\begin{itemize*}
	\item
	{
		\begin{tabular*}{6in}{l@{\extracolsep{\fill}}c}
			\textbf{University of Washington}, Seattle, WA & \textbf{2007 - 2012 (Expected)}\\
		\end{tabular*}
	}
		\begin{tabular*}{6in}{l@{\extracolsep{\fill}}c}
		Ph.D. candidate in Computer Science and Engineering & \\
		Advisor: Prof. Steven M. Seitz & \\
		\end{tabular*}
	\item
	{
		\begin{tabular*}{6in}{l@{\extracolsep{\fill}}c}
			\textbf{Indian Institute of Technology (IIT) Delhi}, India & \textbf{2003 - 2007} \\
		\end{tabular*}
	}
		\begin{tabular*}{6in}{l@{\extracolsep{\fill}}c}
		Bachelor of Technology in Computer Science and Engineering & \\
		(GPA: 9.91/10.0, President's Gold Medal) 
		\end{tabular*}
	\end{itemize*}
%\rule{6.5in}{2pt}
\rule{6.5in}{2pt}
\\
\vspace{0.10in}
{\large \textbf{Publications}}
\begin{itemize*}
\item I. Kemelmacher-Shlizerman, E. Shechtman, R. Garg and S.M. Seitz. \textbf{Exploring Photobios}, \emph{ACM Transactions on Graphics (Proceedings of SIGGRAPH 2011)}
\item R. Garg, D. Ramanan, S.M. Seitz and N. Snavely. \textbf{Where's Waldo: Matching People in Images of Crowds}, \emph{In Proceedings of IEEE Conference on Computer Vision and Pattern Recognition (CVPR), Colorado Springs, June 2011}
\item R. Garg, H. Du, S.M. Seitz and N. Snavely. \textbf{The Dimensionality of Scene Appearance}, \emph{In Proceedings of the IEEE International Conference on Computer Vision (ICCV), Kyoto, Japan, October 2009}
\item N. Snavely, R. Garg, S.M. Seitz and R. Szeliski. \textbf{Finding Paths through the World's Photos}, \emph{ACM Transactions on Graphics (Proceedings of SIGGRAPH 2008)}
\item M. Varma and R. Garg. \textbf{Locally Invariant Fractal Features for Statistical Texture Classification}, \emph{In Proceedings of the IEEE International Conference on Computer Vision (ICCV), Rio De Janeiro, Brazil, October 2007} 
\end{itemize*}
\rule{6.5in}{2pt}
\\
\vspace{0.10in}
{\large \textbf{Experience}}
\begin{itemize*}
\item  
	\begin{tabular*}{6in}{l@{\extracolsep{\fill}}r}
		\textbf{Research Assistant} at \textbf{University of Washington} & \textbf{Fall 2007 - Present} \\
	\end{tabular*}
\\
%\vspace{0.1in}
%\vspace{2pt}
\begin{flushright}
\begin{flushleft}
Ongoing work on developing techniques to aid visualization of scene from a collection of overlapping photos.
\end{flushleft}
\end{flushright}
\item  
	\begin{tabular*}{6in}{l@{\extracolsep{\fill}}r}
		\textbf{Software Engineering Intern} at \textbf{Google Seattle} & \textbf{Summer 2011} \\
	\end{tabular*}
\\
\textbf{Mentor}: Prof. Steven M. Seitz%, Microsoft Research India
\begin{flushright}
\begin{flushleft}
Worked on an Android app that allows the user to take overlapping photos of a scene and browse them in a panorama-like viewer.
\end{flushleft}
\end{flushright}
\item  
	\begin{tabular*}{6in}{l@{\extracolsep{\fill}}r}
		\textbf{Software Engineering Intern} at \textbf{Google Seattle} & \textbf{Spring-Summer 2010} \\
	\end{tabular*}
\\
\textbf{Mentor}: Prof. Steven M. Seitz%, Microsoft Research India
\begin{flushright}
\begin{flushleft}
Implemented the Face Movie feature included in personal photo organizer software Picasa 3.8.
\end{flushleft}
\end{flushright}
\item  
	\begin{tabular*}{6in}{l@{\extracolsep{\fill}}r}
		\textbf{Research Intern} at \textbf{Microsoft Research Redmond} & \textbf{Summer 2008} \\
	\end{tabular*}
\\
\textbf{Mentor}: Dr. Richard  Szeliski%, Microsoft Research India
\begin{flushright}
\begin{flushleft}
Worked on the problem of detecting and matching features across images of man made scenes.
\end{flushleft}
\end{flushright}
%\vspace{0.1in}
%\vspace{2pt}
\item  
	\begin{tabular*}{6in}{l@{\extracolsep{\fill}}r}
		\textbf{Research Engineer} at \textbf{B-Core Software Pvt. Ltd., New Delhi} & \textbf{Summer 2007} \\
	\end{tabular*}
\\
%\vspace{0.1in}
%\vspace{2pt}
\begin{flushright}
\begin{flushleft}
Worked on an application for mobile platforms to decode color codes from an image of a colored sequence (analogous to bar codes)
\end{flushleft}
\end{flushright}

\item  
	\begin{tabular*}{6in}{l@{\extracolsep{\fill}}r}
		\textbf{Research Intern} at \textbf{Microsoft Research India} & \textbf{Summer 2006} \\
	\end{tabular*}
\\
%\vspace{0.1in}
\textbf{Guide}: Dr. Manik Varma%, Microsoft Research India
%\vspace{2pt}
\begin{flushright}
\begin{flushleft}
Worked on developing locally invariant fractal features which allow for statistical analysis and classification of textures
\end{flushleft}
\end{flushright}
\item
	\begin{tabular*}{6in}{l@{\extracolsep{\fill}}r}
		\textbf{Summer Internship} at \textbf{IIT, Delhi} & \textbf{Summer 2005} \\
	\end{tabular*}
\\
%\vspace{0.1in}
\textbf{Guide}: Prof. Subhashis Banerjee%, Department of Computer Science and Engineering, IIT Delhi  
%\vspace{0.05in}
\begin{flushleft}
Worked on detecting vehicles in a video stream that included study of Scale Invariant Feature Transform (SIFT), Principal Component Analysis (PCA) and Generalized Hough Transform. Recipient of prestigious \textbf{SURA} (Summer Undergraduate Research Award), 2005 awarded by Industrial Research and Development Unit (IRD), IIT Delhi 
\end{flushleft}
\end{itemize*}
\rule{6.5in}{2pt}
\\
\vspace{0.10in}
{\large \textbf{Patents}}
\begin{itemize*}
\item Face and Expression Aligned Movies, S.M. Seitz, R. Garg, I. Kemelmacher, US Patent application 61/371,934, pending, filed on Aug 9 2010
\end{itemize*}
\rule{6.5in}{2pt}
\\
\vspace{0.10in}
{\large \textbf{Teaching Experience}}
\begin{itemize*}
\item TA for CSE455 (Undergraduate Computer Vision course) during Winter 2010 at University of Washington.
\item Tutored undergraduate students at University of Washington.
\end{itemize*}
\rule{6.5in}{2pt}
\\
\vspace{0.10in}
{\large \textbf{Talks}}
\begin{itemize*}
\item \textbf{Exploring Photobios} (with Ira Kemelmacher-Shlizerman) 
\\
SIGGRAPH 2011, Vancouver, BC, August 2011.
\item \textbf{Finding Paths through the World's Photos} (with Noah Snavely) 
\\
SIGGRAPH 2008, Los Angeles, CA, August 2008.
\item \textbf{Beyond Photo Tourism: Guided Navigation of Internet Photo Collections} 
\\
Indian Institute of Technology (IIT) Delhi, India, September 2008.
\end{itemize*}
\rule{6.5in}{2pt}
\\
\vspace{0.10in}
{\large \textbf{Academic Achievements}}
\begin{itemize*}
\item Recipient of \textbf{NVIDIA Fellowship} for the year 2009-2010.
\item Recipient of the \textbf{Weil Family Endowed Fellowship} for the year 2007-2008 at the University of Washington
\item Recipient of the \textbf{Clairmont L. Egtvedt Fellowship} awarded by College of Engineering, University of Washington, 2007-2008
\item Awarded  the \textbf{President of India Gold Medal} for securing the highest cumulative GPA among all outgoing B.Tech. students  across all disciplines in 2007 batch, IIT Delhi
\item Secured All India Rank \textbf{7} among 172,000 candidates in IIT-Joint Entrance Examination 2003
\item Secured Rank \textbf{1} in Common Engineering Test (Delhi) 2003, Rank \textbf{18} in All India Engineering Entrance Examination 2003
\item Selected for NTSE (National Talent Search Examination) Scholarship awarded by the Govt. of India
\item Selected for Training Camp, Indian  National Physics Olympiad 2003
\item Awards in many mathematics contests - RMO (Regional Mathematics Olympiad), SMO (Senior Mathematics Olympiad), etc.
\end{itemize*}


\rule{6.5in}{2pt}
\\
%\vspace{0.10in}

%{\large \textbf{Relevant Courses Undertaken}}
%\\
%\vspace{0.1in}
% Analysis and Design of Algorithms,
% Linear Algebra,
% Computer Architecture,
% Data Structures,
% Programming Languages,
% Design Practices in Computer Science,
% Numerical and Scientific Computing,
% Operating Systems,
% Independent Study in Computer Vision,
% Intoduction to Probability and Stochastic Analysis,
% %Computer Networks,
% Digital Image Processing,
% Multivariate Statistical Analysis,
% Computer Graphics
% Computer Vision

%\rule{6.5in}{2pt}
%\\
%\vspace{0.10in}
{\large \textbf{Programming Competitions}}
\begin{itemize*}
	\item Among top 12 finalists in TopCoder Open 2008 Marathon competition, an international event hosted by www.topcoder.com. 
	\item Among top 500 onsite finalists in Google Code Jam 2008, an international programming contest organized by Google.
	\item Among top 50 finalists in Google Code Jam India, 2006 - Programming Contest organized by Google which attracted over 14,000 participants from South East Asia.
	\item Problem setter for International Online Programming Contest organized during Tryst 2007, the technical festival of IIT Delhi. 
	\item Represented IIT Delhi in ACM International Collegiate Programming Contest 2007. 
	\item Regular member of www.topcoder.com - a website hosting global programming competitions, ranked among top 5 in India during undergrad.
\end{itemize*}

\rule{6.5in}{2pt}
%{\large \textbf{Declaration}}
%\\
%\vspace{0.1in}
%I hereby declare that particulars given herein  are true to the best of my knowledge and belief.
%\\
%\vspace{0.1in}
%\textbf{Rahul Garg}
%\\
%\textbf{January 2008}
%\rule{6.5in}{2pt}
\end{document}
