% resume.tex %
%
\documentclass[10pt]{article}
\usepackage{fullpage1}	
\usepackage{atbeginend}
\usepackage{textcomp}
\usepackage{xcolor}
\usepackage[pdftex]{hyperref}
\textheight=9.1in
\pagestyle{empty}
\raggedbottom
\raggedright
\setlength{\tabcolsep}{0in}
\newenvironment{itemize*}%
  {\begin{itemize}%
    \setlength{\itemsep}{0pt}%
    \setlength{\parskip}{0pt}%
	}
  {\end{itemize}}
\newenvironment{description*}%
  {\begin{description}%
   \setlength{\itemsep}{0pt}%
   \setlength{\parskip}{0pt}%
	}
  {\end{description}}

\begin{document}

\begin{tabular*}{7.0in}{l@{\extracolsep{\fill}}r}
\textbf{\large{Rahul Garg}}  & \\
Senior Staff Research Scientist and Manager &  Ph:(206) 708 5975\\
Google Inc. & \href{http://www.rahulgarg.com}{http://www.rahulgarg.com}\\ 
1600 Amphitheatre Parkway, Mountain View, CA 94043 &  Email: \href{mailto:rahul.gargrahul@gmail.com}{rahul.gargrahul@gmail.com}\\
\end{tabular*}
\\
\vspace{0.05in}
\rule{7.0in}{2pt}
\\
\vspace{0.10in}
Over 15 years of experience in computer vision  with a proven track record of converting cutting-edge research to delightful user products and features.

\textbf{Research Interests:} Computational photography, image processing, image based rendering, 3D computer vision.
\rule{7.0in}{2pt}
\\
\vspace{0.10in}
{\large \textbf{Experience}}
\begin{itemize*}
\item
	\begin{tabular*}{6.5in}{l@{\extracolsep{\fill}}r}
		\textbf{Research Scientist} at \textbf{Google} & \textbf{Oct\textquotesingle 13 - Present} \\
	\end{tabular*}
	\begin{itemize*}
	\vspace{0.03in}
	\item[$\circ$]
	\begin{tabular*}{6.2in}{l@{\extracolsep{\fill}}r}
		\textbf{Senior Staff Research Scientist and Manager} at \textbf{Google Labs} & \textbf{Oct\textquotesingle 23 - Present} \\
	\end{tabular*}
	\\
	\vspace{0.03in}
	\item[$\circ$]
	\begin{tabular*}{6.2in}{l@{\extracolsep{\fill}}r}
		\textbf{Staff Research Scientist and Manager} at \textbf{Google Labs} & \textbf{Mar\textquotesingle 21 - Oct\textquotesingle 23} \\
	\end{tabular*}
	\\
	\begin{flushright}
	\begin{flushleft}
		\parbox{6.2in}{
			Computer vision and ML to improve videoconferencing. Grew a ~15 person team, mentored team members, fostered cross-team relationships, led ideation, prototyping, and productization to build features, many of which were launched at CloudNext'23, e.g., \href{https://workspaceupdates.googleblog.com/2023/08/studio-look-google-meet.html}{\underline{Studio Look}} and \href{https://workspace.google.com/blog/product-announcements/duet-ai-in-workspace-now-available}{\underline{Dynamic Tiles}}.
}
	\end{flushleft}
	\end{flushright}
	\vspace{0.03in}
	\item[$\circ$]
	\begin{tabular*}{6.2in}{l@{\extracolsep{\fill}}r}
		\textbf{Staff Research Scientist} at \textbf{GCam, Google Research} & \textbf{Jan\textquotesingle 17 - Mar\textquotesingle 21} \\
	\end{tabular*}
	\\
	\begin{flushright}
	\begin{flushleft}
		\parbox{6.2in}{
			Pioneered cutting-edge computational photography for Google Pixel phones and Google Photos, including \href{https://ai.googleblog.com/2017/10/portrait-mode-on-pixel-2-and-pixel-2-xl.html}{\underline{Portrait Mode}} for \href{https://ai.googleblog.com/2018/11/learning-to-predict-depth-on-pixel-3.html}{\underline{single}} and \href{https://ai.googleblog.com/2019/12/improvements-to-portrait-mode-on-google.html}{\underline{dual-camera}} phones, \href{https://blog.google/products/android/new-android-features-march-2022/\#:~:text=Turn\%20more\%20photos,with\%20Portrait\%20Blur}{\underline{Portrait Blur}} for Google Photos, Auto-focus for Nighsight, and Magic Eraser. These were highlighted  in \href{https://blog.google/inside-google/company-announcements/super-bowl-ad-2023-watch-fixed-on-pixel/}{\underline{Pixel ads}} and made it the \href{https://www.dxomark.com/google-pixel-2-reviewed-sets-new-record-smartphone-camera-quality/}{\underline{top}} \href{https://www.dxomark.com/google-pixel3-camera-review/}{\underline{performing}} smartphone camera for several years in a row.}
	\end{flushleft}
	\end{flushright}
	\vspace{0.03in}
	\item[$\circ$]
	\begin{tabular*}{6.2in}{l@{\extracolsep{\fill}}r}
		\textbf{Senior Research Scientist} at \textbf{Daydream (VR/AR), Google} & \textbf{June\textquotesingle 15 - Jan \textquotesingle 17} \\
	\end{tabular*}
	\\
	\begin{flushright}
	\begin{flushleft}
	Co-founded hand-tracking team and built real-time hand segmentation and tracking using CNNs.
	\end{flushleft}
	\end{flushright}
	\vspace{0.03in}
	\item[$\circ$]
	\begin{tabular*}{6.2in}{l@{\extracolsep{\fill}}r}
		\textbf{Research Scientist} at \textbf{Google Research} & \textbf{Oct\textquotesingle 13 - June \textquotesingle 15} \\
	\end{tabular*}
	\\
	\begin{flushright}
	\begin{flushleft}
	Built real-time hand gesture detection for Google Meet (Hangouts). 
	\end{flushleft}
	\end{flushright}
\end{itemize*}
\vspace{0.03in}
\item  
	\begin{tabular*}{6.5in}{l@{\extracolsep{\fill}}r}
		\textbf{Research Engineer} at \textbf{\href{https://techcrunch.com/2013/10/02/google-acquires-yc-backed-flutter-a-gesture-recognition-technology-startup/}{\underline{Flutter}} (Acquired by Google)} & \textbf{April\textquotesingle 12 - Oct\textquotesingle 13} \\
	\end{tabular*}
\\
\begin{flushright}
\begin{flushleft}
		\parbox{6.5in}{
Startup building low-power real-time hand gesture recognition(Mac app rated 4.5+, among Apple's Best Apps of 2012), one of the first employees.
Sped up recognition by $10\times$,
automated training data collection,
co-invented new state of the art image feature,
developed desktop apps, mobile apps, and browser extensions.
	}
\end{flushleft}
\end{flushright}
\item  
	\begin{tabular*}{6.5in}{l@{\extracolsep{\fill}}r}
		\textbf{Software Engineering Intern} at \textbf{Google Seattle} & \textbf{Mar\textquotesingle 10 - Sep\textquotesingle 10, June\textquotesingle 11 - Sep\textquotesingle 11} \\
	\end{tabular*}
\\
\begin{flushright}
\begin{flushleft}
	Successfully implemented and launched the \href{http://googlephotos.blogspot.com/2010/08/picasa-38-face-movies-picnik.html}{\underline{Face Movie}} feature for Picasa as an intern, receiving widespread press coverage and 1.5M+ \href{http://www.youtube.com/watch?v=fLQtssJDMMc}{\underline{YouTube}} views. Published at SIGGRAPH and 3DV.
	\end{flushleft}
\end{flushright}
\item  
	\begin{tabular*}{6.5in}{l@{\extracolsep{\fill}}r}
		\textbf{Research Intern} at \textbf{Microsoft Research} & \textbf{June\textquotesingle 08 - Sep\textquotesingle 08, May\textquotesingle 06 - July\textquotesingle 06} \\
	\end{tabular*}
\begin{flushright}
\begin{flushleft}
	Conducted research on feature matching, 3D reconstruction, and texture classification. Published at ICCV.
\end{flushleft}
\end{flushright}
%\vspace{0.1in}
%\vspace{2pt}
%\item  
%	\begin{tabular*}{7.0in}{l@{\extracolsep{\fill}}r}
%		\textbf{Research Engineer} at \textbf{B-Core Software Pvt. Ltd., New Delhi} & \textbf{June 2007 - Aug 2007} \\
%	\end{tabular*}
%\\
%\vspace{0.1in}
%\vspace{2pt}
%\begin{flushright}
%\begin{flushleft}
%Worked on an image segmentation application for mobile platforms to decode color codes from an image of a colored sequence (analogous to bar codes).
%\end{flushleft}
%\end{flushright}

%\item  
%	\begin{tabular*}{7.0in}{l@{\extracolsep{\fill}}r}
%		\textbf{Research Intern} at \textbf{Microsoft Research India} & \textbf{May 2006 - July 2006} \\
%	\end{tabular*}
%\\
%\vspace{0.1in}
%\vspace{2pt}
%\begin{flushright}
%\begin{flushleft}
%  Developed locally invariant fractal features for statistical analysis and classification of textures. Published at ICCV 2007.
%\end{flushleft}
%\end{flushright}
\end{itemize*}
\rule{7.0in}{2pt}
\\
\vspace{0.10in}
{\large \textbf{Education}}

	\begin{itemize*}
	\item
	{
		\begin{tabular*}{6.5in}{l@{\extracolsep{\fill}}c}
			\textbf{University of Washington}, Seattle, WA & \textbf{2007 - 2012}\\
		\end{tabular*}
	}
		\begin{tabular*}{6.5in}{l@{\extracolsep{\fill}}c}
                \emph{Ph.D. in Computer Science and Engineering} & \\
                \textbf{Advisor}: Prof. Steven M. Seitz & \\
		\end{tabular*}
	\item
	{
		\begin{tabular*}{6.5in}{l@{\extracolsep{\fill}}c}
			\textbf{Indian Institute of Technology (IIT) Delhi}, India & \textbf{2003 - 2007} \\
		\end{tabular*}
	}
		\begin{tabular*}{6.5in}{l@{\extracolsep{\fill}}c}
                  \emph{Bachelor of Technology in Computer Science and Engineering} & \\
                \textbf{GPA: 9.91/10.0, President's Gold Medal}\\
                Highest GPA amongst all outgoing B.Tech. students across all majors\\
                \textbf{All India Rank 7} among 172000 candidates in IIT-Joint Entrance Exam. 2003
		\end{tabular*}
	\end{itemize*}
\rule{7.0in}{2pt}
\\
\vspace{0.10in}
{\large \textbf{Honors and Awards}}
\begin{itemize*}
\item Publications at top-tier vision and graphics conferences (CVPR, SIGGRAPH, ICCV, ECCV). 10$+$ patents. 
\item Fellowships: NVIDIA Fellowship (2009-10), Clairmont L. Egtvedt Fellowship (2007-08), Weil Family Endowed Fellowship (2007-08), NTSE Scholarship by Govt. of India (2001).
\item Programming Contests: \textbf{Top 12} in TopCoder Open 2008 Marathon competition (\textbf{international} event), \textbf{Top 50} in Google Code Jam India 2006
\end{itemize*}
%\rule{7.0in}{2pt}
\rule{7.0in}{2pt}
\\
\vspace{0.10in}
{\large \textbf{Selected Publications}}
\begin{itemize*}
\item Y. Zhang, N. Wadhwa, S. Orts-Escolano, C. H\"ane, S. Fanello, R. Garg, \textbf{Du$^2$Net: Learning Depth Estimation from Dual-Cameras and Dual-Pixels}, \emph{ECCV, 2020.} ({\color{red} Oral. Basis for Portrait Mode on Pixel 4.})
\item C. Herrmann, R. S. Bowen, N. Wadhwa, R. Garg, Q. He, J.T. Barron, R. Zabih, \textbf{Learning to Autofocus}, \emph{CVPR, 2020.} ({\color{red} Basis for low light autofocus on Pixel 5})
\item R. Garg, N. Wadhwa, S. Ansari, J.T. Barron, \textbf{Learning Single Camera Depth Estimation using Dual-Pixels}, \emph{ICCV, 2019.} ({\color{red}Oral. Basis for Portrait Mode on Pixel 3.})
\item S. Ansari, N. Wadhwa, R. Garg, J. Chen, \textbf{Wireless Software Synchronization of Multiple Distributed Cameras}, \emph{ICCP, 2019.} ({\color{red} Used for training data collection for various Pixel features.})
\item N. Wadhwa, R. Garg, D.E. Jacobs, B.E. Feldman, N. Kanazawa, R. Carroll, Y. Movshovitz-Attias, J.T. Barron, Y. Pritch, M. Levoy, \textbf{Synthetic Depth-of-Field with a Single-Camera Mobile Phone.}, \emph{SIGGRAPH, 2018.} ({\color{red} Basis for Portrait Mode on Pixel 2.})
\item I. Kemelmacher-Shlizerman, E. Shechtman, R. Garg and S.M. Seitz. \textbf{Exploring Photobios}, \emph{SIGGRAPH, 2011}. Featured on SIGGRAPH cover. ({\color{red} Basis for Google Picasa Face Movies feature.})
\item N. Snavely, R. Garg, S.M. Seitz and R. Szeliski. \textbf{Finding Paths through the World's Photos}, \emph{SIGGRAPH, 2008.} ({\color{red} Basis for new features in Microsoft's Photosynth.})
\end{itemize*}
Complete list on \href{https://scholar.google.com/citations?user=0eQjcEEAAAAJ}{\underline{Google Scholar}}.
\rule{7.0in}{2pt}
\end{document}
